\documentclass[12pt]{book}
\usepackage[T1]{fontenc}
\usepackage[T2A]{fontenc}
\usepackage[utf8]{inputenc}
\usepackage[english, russian]{babel}
\usepackage{amsmath}
\textwidth=371pt
\textheight=608pt
\pagestyle{empty}
\begin{document}

\small{ \slshape{ \noindent 278     \hfill Лекция 19. Формула и ряд Тейлора} \upshape} \normalsize
\hrule\bigskip

 $$ b)\quad \lim_{x\to +0} \Big(\ctg2x \Big)^\frac{1}{\ln x}$$ \\
\bfseries  {Лекция 19. Формула и ряд Тейлор} \mdseries  \\

\fbox{%
        \parbox{318pt}{%
Многочлен Тейлора. Формула Тейлора. Ряд тейлора. Формулы и ряды Тейлора для некоторых элементарных функций и использование их для вычисления пределов и в приближённых вычислениях.
        }%
} \\


Формула и ряд Тейлора\footnote[9]{Б.Тейлор(1685-1731) - английский математик.} относятся к числу важнейших понятий математического анализа. Они широко используются в математике. В частности, в этом семестре мы будем использовать их при исследовании функций, приближённом вычислении значений функций, численном решении уравнений и оценке возникающих при этом ошибок.\\

\bfseries 19.1. Многочлен Тейлора \mdseries \\

Пусть функция $y = f(x)$ $n$ раз дифференцируема в некотором интервале, содержащем точку $x_{0}$. Построим многочлен степени $m<n$ \\
$$P_{m}(x)=C_{0}+C_{1}(x-x_{0})-C_{2}(x-x_{0})^2+ \cdots +C_{m}(x-x_{0})^m=\quad  (19.1) $$
$$=\sum\limits_{k=0}^{m} C_{k}(x-x_{0})^k,$$

\noindent коэфиценты которого определим из условия совпадения значений этого многочлена
 и его m производных с соответствующими значениями функций $y = f(x)$ и её
 производных в точке $x_{0}$.
\begin{equation} \tag{19.2}
\begin{cases} 
$$P_{m}(x_{0})=C_{0}=f(x_{0}),$$\\
$$P_{m}^{'}(x_{0})=C_{1}=f^{'}(x_{0}),$$\\
$$P_{m}^{''}(x_{0})=2 \cdot 1 \cdot C_{2}=f^{''}(x_{0}),$$\\
$$P_{m}^{'''}(x_{0})=3 \cdot 2 \cdot 1 \cdot C_{3}=f^{'''}(x_{0}),$$\\
$$\cdots\cdots\cdots\cdots\cdots\cdots\cdots\cdots\cdots\cdots\cdots\cdots\cdots\cdots\cdot\cdot $$\\   
$$P_{m}^{(k)}(x_{0})= k \cdot (k-1)\dots 2 \cdot 1 \cdot C_{k}=f^{(k)}(x_{0}),$$\\
$$\cdots\cdots\cdots\cdots\cdots\cdots\cdots\cdots\cdots\cdots\cdots\cdots\cdots\cdots\cdot\cdot$$\\
$$P_{m}^{(m)}(x_{0})= m \cdot (m-1)\dots 2 \cdot 1 \cdot C_{m}=f^{(m)}(x_{0}).$$\\
\end{cases}
\end{equation}



\small{\slshape{ \noindent Лекция 19. Формула и ряд Тейлора     \hfill 279} \upshape} \normalsize
\hrule\bigskip
Члены в производных от многочлена (19.1), содержащие множитель $(x-x_{0})$ в точке $x_{0}$, равны нулю. Производные порядка выше $m$ от многочлена $m$-й степени также равны нулю. Из (19.2) имеем 
\begin{equation} \tag{19.3}
 C_{k}=\frac{f^{(k)}(x_{0})}{k!},k=0,1,\dots,m.
\end{equation}

В (19.3) положено $0!=1, f^{(0)}(x_{0})=f(x_{0}).$ 

Следовательно, искомый многочлен (19.1) с коэффициентами $C_{k}$, определяемыми по формуле (19.3), будет
$$P_{m}(x)=f(x_{0})+f^{'}(x_{0})(x-x_{0})+  \cdots + \frac{f^{(m)}(x_{0})}{m!}(x-x_{0})^{m}= \quad (19.4)$$
$$=\sum\limits_{k=0}^{m}\frac{f^{(k)}(x_{0})}{k!}(x-x_{0})^{m}.$$

\textsc{Определение} 19.1.\textit{ Многочлен (19.4) называется $m$-м многочленом Тейлора функции $f(x)$ по степеням} $(x-x_{0}).$

Если функция $f(x)$ сама по себе является многочленом степени $m$, то запись ее в виде (19.4) всегда возможна и озночает лишь представлени данного многочлена по степеням разности $(x-x_{0})$.\\

\textsc{Пример} 19.1.\textit{ Представить функцию $f(x)=x^{2}$ в виде многочлена Тейлора по степеням $(x-3)$.} \\

Р\,е\,ш\,е\,н\,и\,е: Имеем $f(3)=9, f^{'}(3)=6, f^{''}(3)=2.$ Все производные выше второго порядка от $f(x)=x^{2}$ равны нулю. Следовательно, многочлен Тейлора при $x_{0}=3$ для $f(x)=x^2$ имеет вид
$$P_{2}(x)=9+6(x-3)+(x-3)^{2}.$$
Раскрыв скобки и приведя подобные члены, очевидно, получим данную функцию $f(x)=x^{2}.$ \\

П\,р\,и\,м\,е\,р 19.2.\textit{ Пусть $f(x)=(a+x)^{m}$ и $x_{0}=0$.} \\

Тогда по (19.4) 
$$P_{m}(x)=\sum\limits_{k=0}^{m} \frac{((a+x)^{m})_{x=0}^{(k)}}{k!}x^{k}=(a+x)^{m},$$
где
\begin{equation*} \begin{split}
&((a+x)^{m})^{(k)}=m(m-1)...(m-k+1)(a+x)_{x=0}^{m-k}= \\
&=m(m-1)...(m-k+1)a^{m-k}
\end{split}
\end{equation*}

\end{document}